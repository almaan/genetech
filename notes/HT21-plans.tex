% Created 2021-06-16 Wed 19:36
% Intended LaTeX compiler: pdflatex
\documentclass[11pt]{article}
\usepackage[utf8]{inputenc}
\usepackage[T1]{fontenc}
\usepackage{graphicx}
\usepackage{grffile}
\usepackage{longtable}
\usepackage{wrapfig}
\usepackage{rotating}
\usepackage[normalem]{ulem}
\usepackage{amsmath}
\usepackage{textcomp}
\usepackage{amssymb}
\usepackage{capt-of}
\usepackage{hyperref}
\author{Alma Andersson}
\date{\today}
\title{Planning CB2040 and BB2250}
\hypersetup{
 pdfauthor={Alma Andersson},
 pdftitle={Planning CB2040 and BB2250},
 pdfkeywords={},
 pdfsubject={},
 pdfcreator={Emacs 26.1 (Org mode 9.4.4)}, 
 pdflang={English}}
\begin{document}

\maketitle
\tableofcontents



\section{Infrastructure}
\label{sec:org99ef63a}
\subsection{github}
\label{sec:org0a82b05}
\begin{itemize}
\item The github page will be restructured to contain the following three branches:
\begin{itemize}
\item master : instructions with instructions regarding how to download (clone)
the correct repo and run the Docker image we will assemble for them.
\item cb2040 : all the content and instructions for the cb2040 students
\item bb2255 : all the content and instructions for the bb2255 students
\end{itemize}
\item By implementing this structure we will minimize the risk to encounter
scenarios where students gets confused about which labs they are supposed to
work with (which happened last year, although this was due to them not
reading the instructions)
\item Create a Docker image with all the necessary packages and backend-libraries
(system installations) that might be required to execute the labs.
\item Person in charge of github chagnes : Alma
\end{itemize}

\subsection{Docker}
\label{sec:orgb05764c}
\begin{itemize}
\item Handling Docker images and containers can be a bit messy, so preferably we
only want to produce a single image that we distribute by the start of each
course session. Therefore, it's \textbf{\textbf{important}} that all the labs are
finalized at least \textbf{\textbf{one week}} before we start so I can compile the image
and make sure the containers run smoothly. I also need to  be given a list
of all packages that are being used in the lab each person is responsible
for, both packages that are imported (\texttt{library(package)}) as well as packages
from which specific functions are imported by namespace calling (\texttt{package::fun}).

\item We'll be using the a \uline{rocker} extension where rstudio is included, see
\texttt{https://github.com/rocker-org/rocker}. This should give a fairly seamless
user-experience for the students, as long as they manage to install Docker
on their computers. This allows the students to run rstudio in their
web-browser (locally hosted).

\item The students will all be working with the same image, but hosting their own
containers (as one does with Docker). However, there is one big caveat to
using these transient containers, being that if they are removed or damaged
all their data/progress is lost - also, it's often hard (for someone
inexperienced) to access the files/directories inside a container from the
host-system; making it troublesome for them to upload their pdf-files. I've
evaluated some different options, and believe the easiest way to circumvent
this issue is to use so called \uline{bind-mounts} (see
\texttt{https://docs.docker.com/storage/bind-mounts/}). The bind-mounts are easier
to handle than volumes and tmpfs mounts are (obviously) only active as long
as the container is running.

The students wouldn't have to do any extra-work with the bind mounts,
attaching them to the containers is something that will be included in the -
by us - provided in the \texttt{docker run...} command used to create containers.

\item Person in charge of Docker-related work: Alma
\end{itemize}

\section{Exercises}
\label{sec:org88cab44}

\begin{itemize}
\item In general we will try to harmonize the exercises to create a more
continuous narrative, where the students focus on a single tissue type in
Lab 2-4 (Lab 1 is the introduction to R and does not include any analysis).

\item The tissue type of interest will \uline{breast cancer} data, which is preferable
for several reasons, two of them being:

\begin{enumerate}
\item \uline{Data accessibility} : there are plenty of public bulk RNA-seq, single cell
and Visium data available, which can be easily accessible and used in our labs.
\item \uline{Relevancy} : it's easier to engage the students in any form of work if
the questions they are working with; casting the exercises as analysis
and characterization of a well-known disease will hopefully have a
positive effect on interest.
\end{enumerate}
\end{itemize}

\subsection{Lab 1}
\label{sec:org35b6022}
\begin{itemize}
\item Lab 1 will remain more or less unchanged as it focuses on an introduction to
R. Only minor changes w.r.t. language and grammar will be adjusted.
\item Person in charge of revising Lab 1 is : Alma
\item Person in charge of grading Lab 1 : Alma
\end{itemize}

\subsection{Lab 2}
\label{sec:org7fc0a75}
\begin{itemize}
\item Following the discussion Lab 2 will be revised putting more emphasis on bulk
RNA-seq, and less focus on the GWAS part.
\item My suggestion is to use one of the breast cancer data sets that can be found
in the TCGA (The Cancer Genome Atlas). This data has plenty of meta-data
associated with it, allowing us to conduct analyses that produces
Kaplan-Meier plots representing the survival curves of different strata etc.
This also, to some extent, highlights how it's currently more feasible to
collect bulk RNA-seq from a large cohort of patients, while single cell and
spatial RNA-seq are often more commonly used for in-depth analysis of a
small set of patients. The idea would also be to highlight some of the
weaknesses with bulk RNA-seq, setting the stage for Lab 3 (single cell
RNA-seq).
\item Person in charge of rewriting Lab 2 : Alma
\end{itemize}

\subsection{Lab 3}
\label{sec:orgec907e8}
\begin{itemize}
\item Lab 3 needs some minor updates, to be more consistent with the new
narrative (breast cancer focus). This would mainly include:
\begin{itemize}
\item Change of data set to work with, proposed data is Alex Swarbrick's HER2 data.
\item Clearly highlighting the differences between single cell RNA-seq and
bulk, i.e., putting emphasis on what sets them apart and what information
that can be gained from single cell RNA-seq (hopefully mention
intraatient heterogeneity).
\item Also clearly highlighting what information we can't obtain from the
single cell RNA-seq study, but which spatial transcriptomics may provide
(setting the stage for Lab 2).
\end{itemize}

\item This lab is somewhat tricky in the sense that it should be standalone from
Lab 2 (since students of BB2255 aren't doing that lab), but also fit into
the story.

\item Person in charge of rewriting Lab 3 : Sami
\end{itemize}

\subsection{Lab 4}
\label{sec:org86979e5}
\begin{itemize}
\item Lab 4, just as Lab 3, will not have to go through any major changes, but
just update the data set that we will be working with.
\item As a suggestions we can work with the 10x publicly available breast cancer
data sets.
\item Ideally we will also map the single cell data (used in Lab 3) onto the
spatial transcriptomics data, which closes the story in a neat way.
\item Important is that we highlight how the two different modalities (single
cell vs. spatial) complement each other.

\item Person in charge of rewriting Lab 3 : Alma
\end{itemize}
\end{document}